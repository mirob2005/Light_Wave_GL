\chapter{CONCLUSION}

\section{GENERAL OBSERVATIONS}
Most importantly, indirect shadows are an expensive luxury.  If the targeted machine can support the use of many indirect shadow maps, at least 10 but preferably 100's if not more, it is likely best to just ignore indirect shadows altogether.  As discussed, indirect shadows are the most expensive procedure, but contributes the least to the final overall scene.  Also, due to the low frequency nature of indirect shadows, most observers will not notice their absence and if necessary, they can be faked easier than being actually computed.  As computing technology advances further, such a luxury will become more and more commonplace but in the meantime a good motto would be: If you can't do it right, leave it out.

Indirect lighting, however, adds to the overall realism of the rendered scene.  Modern day computers should be able to render indirect lighting in real-time with acceptable quality and performance.  Should the scene need to be rendered on a limited performance machine, this too can be faked, however, in most cases this would be overkill.

\section{IMPLEMENTATION SPECIFIC FINAL THOUGHTS}
Light Wave is a technique of estimating indirect lighting in a scene in real-time.  It is able to do through the use of virtual point lights or VPL's structured outward from the primary light source in such a way so that the direct light can wrap around objects and hit surfaces that may not be directly visible from the primary light source.  These VPL's are structured in hemispheres around the primary light source with extended viewing capabilities.  These attributes make the overall flow of the light in the scene to resemble a wave flowing outward from the primary light source in all directions and wrapping around obscuring objects.  Therefore, we approximate indirect lighting using only direct lighting calculations from upwards of thousands of light sources.  We are only concerned with where each light ray terminates and not what happens afterwards thus simplifying the calculation.  By doing this we avoid the infinite number of iterations from using a Neumann series to calculate the infinite bouncing of light throughout a scene.

An aim of Light Wave is that it can be scalable.  The implementation allows for the adjustment of parameters which allow the user to increase/decrease performance and thus decrease/increase quality depending on computing power and required quality.  The implementation will also allow for more accurate renderings provided computing advancements.  The default parameter implementation as detailed in prior sections leads to 55 frames per second on the current machine (see section \ref{sec:impdetails} for specifics).  It is assumed that some machines will perform slower than ours, therefore, we provided some suggestions on how to modify the parameters in order to maximize performance with limited quality impact in section \ref{sec:finalAnalysis}.  For machines that can handle the defaults and crave more realistic renderings, the first thing to increase would be the number of indirect shadow maps.

Therefore, this implementation was successful in it's goal of approximating indirect illumination in real-time using the GPU.  It achieves fairly realistic rendered scenes that are fully dynamic allowing the user to move objects and the light in real-time.  It also proves portable with adjustable parameters allowing it to scale to the computing capabilities of the current machine.

Future advances in computing power would allow for specific changes in parameters to allow for more realistic renderings.  These include the use of additional VPL's, use of higher resolution, and the use of additional indirect shadow maps in order to render ultra-realistic indirect shadowing.
