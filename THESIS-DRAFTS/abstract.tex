\begin{abstract}

With the growth of computers and technology, so to has grown the desire to accurately recreate our world using computer graphics.  However, our world is very complex and in many ways beyond our comprehension.  Therefore, in order to perform this task, we must consider multiple disciplines and areas of research including physics, mathematics, optics, geology, and many more to at the very least approximate the world around us.  The applications of being able to do this are plentiful as well, including the use of graphics in entertainment such as movies and games, science such as in weather forecasts or simulations and medicine with body scans, or use in architecture, design, and many other areas.  In order to recreate the world around us, an important task is to accurately recreate the way light travels and affects the objects we see.  Rendering lighting has been a heavily researched area since the 1970's and has gotten more sophisticated over the years.  Until recent developments in technology, realistic lighting of scenes has only been achievable offline taking seconds to hours or more to create a single image, however, due to advances in graphics technology, realistic lighting can be done in real-time.  To achieve real-time rendering, we must make trade offs between scientific accuracy and performance, but as will be discussed later, scientific accuracy may not be necessary after all.

\end{abstract}
