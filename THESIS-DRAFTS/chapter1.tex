\chapter{BACKGROUND}

\section{LIGHT AS A PARTICLE AND A WAVE}
Before discussing current topics of research in the field on realistic light rendering in computer graphics, we first need a basic understanding of optics, a branch of physics that studies the behavior and properties of light.  Prior to the nineteenth century, light was regarded as a stream of particles and with this particle theory of light in mind, certain light phenomena such as refraction and reflection could be explained.  However in 1801, the first evidence of light acting as a wave was found when light rays were found to interfere with one another.  Later, in 1873, light was compared to a form of a high-frequency electromagnetic wave.  In 1905, Einstein proposed a theory that explained the photoelectric effect (ejection of electrons from a metal surface when hit by light) that used the concept of quantization and stated that the energy of light waves are present in particles called photons which are quantized.  \cite{Serway2004}  Therefore, light can have the behavior of a wave and a particle.  These findings are important to consider in our research to accurately depict light as it travels throughout a scene.  As will be discussed in the related work section, many lighting techniques use the idea that light consists of particles in order to simulate proper lighting conditions.  Some even use the terminology of “photons” such as the lighting technique photon mapping.  This paper, however, will try to explore a hybrid approach that considers light as both a particle and as a wave.

\section{TERMINOLOGY}
Lastly, before jumping into the related work section, we must cover some terminology used in current papers covering lighting in computer graphics.  Lighting or illumination, is broken down into different components.  Direct illumination is the rendering of a scene as it would appear with the light rays from the light sources terminating at the first surface it hits.  Scenes made with this type of illumination are high performance but low realism due to attributes such as hard shadows, no reflection or refraction, and the fact that only surfaces in view of the light source will be illuminated with everything else black.  In order to add realism, indirect illumination needs to be added to our prior directly illuminated scene.  Indirect illumination considers what happens to the light after it comes into contact with a surface.  Indirect illumination takes into account the reflection and refraction of light and produces realistic scenes depending on the extent of the indirect illumination captured.  Indirect illumination produces the illumination seen on surfaces that are not directly viewable by the light source such as behind another object.  The performance of indirect illumination varies greatly due to the extent of indirect illumination captured.  Noting that this indirect illumination can be recursively calculated to infinity, indirect illumination can be expanded to a Neumann series and would be calculated by calculating the sum of light incident on a surface due to the reflection of light n-times all the way up to infinity.  This fact will be explored further in the related works section, but the main idea is that the performance of calculating the indirect illumination depends on how large n is along with other factors discussed later, but is typically much slower than calculating direct illumination.  Once we have direct illumination and indirect illumination calculated, we can say we have calculated the global illumination of the scene.  This paper's technique will try to exploit the high performance of direct illumination by trying to approximate the indirect illumination of a single light source by using the direct illumination of multiple virtual light sources.
