\chapter{BACKGROUND}

\section{LIGHT AS A PARTICLE AND A WAVE}
\paragraph{}
Before discussing current topics of research in the field of realistic light rendering in computer graphics, we first need a basic understanding of optics, a branch of physics that studies the behavior and properties of light.  Prior to the nineteenth century, light was regarded as a stream of particles and with this particle theory of light in mind, certain light phenomena such as refraction and reflection could be explained.  However in 1801, the first evidence of light acting as a wave was found when light rays were found to interfere with one another.  Later, in 1873, light was compared to a form of high-frequency electromagnetic wave.  In 1905, Einstein proposed a theory that explained the photoelectric effect (ejection of electrons from a metal surface when hit by light) that used the concept of quantization and stated that the energy of light waves is quantized in particles called photons. Quantization is the idea of constraining elements that are continuous into discrete sets. \cite{Serway2004}  Therefore, light can have the behavior of a wave and a particle.  These findings are important to consider in our attempt to accurately depict light as it travels throughout a scene.  As will be discussed in the related work section, many lighting techniques use the idea that light consists of particles in order to simulate proper lighting conditions.  Some even have used the term photons such as in the lighting technique photon mapping.  This paper, however, will try to explore a hybrid approach that considers light as both a particle and as a wave.

\section{TERMINOLOGY}
\paragraph{}
Lastly, before jumping into the related work section, we must cover some basic terminology used in current papers covering lighting in computer graphics.  Lighting or illumination, is broken down into different components.  Direct illumination is the rendering of a scene as it would appear if the light rays from the light sources terminated or ended at the first surface they hit.  Scenes made with this type of illumination are high performance but low realism due to attributes such as hard shadows, no reflection or refraction, and the fact that only surfaces in view of the light source are illuminated with everything else black.  In order to make the scene more realistic, we need to add indirect illumination.  Indirect illumination considers the behavior of light after it hits a surface.  It features the reflection and refraction of light and produces realistic scenes depending on the extent of indirect illumination captured.  It illuminates the surfaces that are not directly visible by the light source such as behind another object.  Light can be interpreted as bouncing through a room infinitely until the energy of the light reaches zero and is completely absorbed by the surfaces in the room.  Noting this, indirect illumination can be recursively calculated to infinity and therefore can be expressed using a Neumann series where we calculate the sum of light hitting a surface point due to the reflection of light from other surfaces $n$-times with $n$ approaching infinity.  Then the performance of calculating the indirect illumination depends on how large $n$ is along with other factors discussed later, but is typically much slower than calculating direct illumination.  Once we have direct illumination and indirect illumination calculated, we can say that we have the global illumination of the scene.  This paper's goal is to exploit the high performance of direct illumination by trying to approximate the indirect illumination of a single light source by using the direct illumination of multiple virtual light sources.
